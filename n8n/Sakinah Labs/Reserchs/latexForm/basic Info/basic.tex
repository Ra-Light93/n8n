\documentclass[12pt,a4paper]{article}
\usepackage[utf8]{inputenc}
\usepackage[T1]{fontenc}
\usepackage[english]{babel}
\usepackage[margin=2cm]{geometry}
\usepackage{titlesec}
\usepackage{xcolor}
\usepackage{enumitem}
\usepackage{setspace}
\usepackage{lmodern}
\usepackage{helvet}
\usepackage{graphicx}

% Farben definieren
\definecolor{primary}{RGB}{44, 62, 80}
\definecolor{secondary}{RGB}{52, 152, 219}
\definecolor{lightblue}{RGB}{236, 240, 241}
\definecolor{lightgreen}{RGB}{234, 250, 241}
\definecolor{darkblue}{RGB}{41, 128, 185}
\definecolor{darkgray}{RGB}{52, 73, 94}

% Schriftarten einstellen (für pdflatex)
\renewcommand{\familydefault}{\sfdefault} % Sans-serif Schrift wie Arial
\setstretch{1.6}

% Titelformatierung
\titleformat{\section}
{\normalfont\Large\bfseries\color{primary}}
{}{0pt}
{\hspace{-0.4pt}\rule{\textwidth}{0.4pt}\\\vspace{5pt}}

\titleformat{\subsection}
{\normalfont\large\bfseries\color{primary}}
{\thesubsection}{0.5em}{}

\titleformat{\subsubsection}
{\normalfont\normalsize\bfseries\color{darkgray}}
{\thesubsubsection}{0.5em}{}

% Listenformatierung
\setlist[itemize]{
    leftmargin=*,
    label={},
    before=\vspace{0.5\baselineskip},
    after=\vspace{0.5\baselineskip}
}

\newenvironment{coloreditem}
{\begin{list}{}{
    \setlength{\leftmargin}{0pt}
    \setlength{\itemindent}{0pt}
    \setlength{\listparindent}{0pt}
    \setlength{\topsep}{6pt}
    \setlength{\partopsep}{0pt}
    \setlength{\parsep}{0pt}
    \setlength{\itemsep}{6pt}}
    \renewcommand{\makelabel}[1]{}
}
{\end{list}}

\newcommand{\important}[1]{\textbf{\color{darkblue}#1}}

% Hinweis-Box
\newcommand{\note}[1]{
    \vspace{10pt}
    \begin{center}
    \fbox{
        \parbox{0.9\textwidth}{
            \centering
            \textbf{\color{primary}#1}
        }
    }
    \end{center}
    \vspace{10pt}
}

% Box für farbige Listenelemente
\newenvironment{coloredbox}{
    \begin{list}{}{
        \setlength{\leftmargin}{15pt}
        \setlength{\itemindent}{-15pt}
        \setlength{\listparindent}{0pt}
        \setlength{\topsep}{6pt}
        \setlength{\partopsep}{0pt}
        \setlength{\parsep}{0pt}
        \setlength{\itemsep}{6pt}}
        \renewcommand{\makelabel}[1]{}
    }
    {\end{list}}

\begin{document}

\begin{center}
    {\Huge\bfseries\color{primary}Social Media Strategy: \\[5pt] Automated Islamic Content}
    
    \vspace{20pt}
    
    \note{This strategy is designed for an automated workflow, focusing on scalable and template-based content for daily posting on Reels, TikTok, and YouTube Shorts.}
\end{center}

\section*{1. Content Strategy: The Four Pillars}
Content should be consistent in theme but varied in format. All videos must be short (15-60 seconds), visually appealing, and include clear, easy-to-read captions.

\subsection*{Pillar 1: Daily Reminders (High Frequency)}
\begin{coloredbox}
    \item \important{What:} Short, impactful verses from the Quran or a Hadith.
    \item \important{Format:} Display text over high-quality, serene stock footage (nature, mosques, calligraphy) or a short, powerful Quran recitation with on-screen translation.
    \item \important{Goal:} Easily shareable, inspirational content providing immediate spiritual value.
\end{coloredbox}

\subsection*{Pillar 2: Educational Snippets (Medium Frequency)}
\begin{coloredbox}
    \item \important{What:} Explain a single concept, the story behind a verse, or a simple Sunnah practice.
    \item \important{Format:} Q\&A style ("What is 'Alhamdulillah'?") or listicles ("3 Sunnahs for a Productive Morning").
    \item \important{Goal:} Position the account as a source of clear, reliable knowledge.
\end{coloredbox}

\subsection*{Pillar 3: Inspirational Storytelling (Medium Frequency)}
\begin{coloredbox}
    \item \important{What:} Short stories from the Seerah, the lives of the Sahaba, or Islamic history.
    \item \important{Format:} A voiceover narrates the story accompanied by relevant visuals (illustrations, stock clips). Focus on the core moral.
    \item \important{Goal:} Create an emotional connection and make lessons memorable.
\end{coloredbox}

\subsection*{Pillar 4: Community Engagement (Low Frequency)}
\begin{coloredbox}
    \item \important{What:} Content that encourages interaction.
    \item \important{Format:} Ask a question ("What verse is bringing you peace this week?") or show a brief "behind the scenes" of the content creation.
    \item \important{Goal:} Foster a sense of community and active participation.
\end{coloredbox}

\section*{2. Posting Frequency \& Schedule}
Consistency is more important than volume. This schedule is ideal for an automated system.

\begin{coloredbox}
    \item \important{TikTok:} 2-3 times per day.
    \item \important{Instagram Reels:} 1-2 times per day.
    \item \important{YouTube Shorts:} 1 time per day.
\end{coloredbox}

\subsection*{Recommended Daily Automated Schedule:}
\begin{coloredbox}
    \item \important{Morning (7-9 AM):} A \important{Daily Reminder}.
    \item \important{Afternoon (1-3 PM):} An \important{Educational Snippet} or \important{Inspirational Story}.
    \item \important{Evening (7-9 PM):} A second \important{Daily Reminder}.
\end{coloredbox}

\section*{3. Audience Growth \& Engagement}
Strong growth comes from providing value and building community.

\begin{coloredbox}
    \item \important{Strategic Hashtags:} Use a mix of broad (\#islam, \#muslim), niche (\#islamicreminder, \#deen), and platform-specific (\#fyp, \#reels) hashtags in every post.
    \item \important{Engage with Comments:} Manually respond to comments and questions daily. This human touch is crucial and cannot be fully automated.
    \item \important{Calls to Action (CTA):} End videos with a clear, automated CTA like "Follow for daily Islamic reminders" or "Share this with someone who needs it."
\end{coloredbox}

\section*{4. How to Avoid Problems \& Build Trust (Crucial for Automation)}
This niche demands the highest level of responsibility.

\begin{coloredbox}
    \item \important{Absolute Accuracy:} Your automation scripts MUST pull from a pre-vetted and 100\% accurate database of Quranic verses and Hadith. For Hadith, include the source (e.g., Sahih al-Bukhari, 1) in the video text itself.
    \item \important{Respectful Content:}
    \begin{itemize}
        \item \important{Audio:} Use only high-quality, instrument-free nasheeds or human voice/natural sounds. Your media library must be strictly curated.
        \item \important{Visuals:} Your visual asset library must only contain appropriate and respectful imagery (nature, architecture, abstract patterns, calligraphy).
    \end{itemize}
    \item \important{Avoid Controversial Topics:} Configure your automation to only pull from universally accepted and inspirational topics. Avoid complex fiqh or sectarian subjects.
    \item \important{Handling Negativity:}
    \begin{itemize}
        \item \important{Manual Review:} Plan for a daily manual review of comments. Do not engage with trolls; simply delete and block.
        \item \important{Correct Politely:} If someone posts misinformation, correct them politely with sources. Do not get into arguments.
    \end{itemize}
\end{coloredbox}

\section*{5. Platform-Specific Content Strategy}
This section provides a detailed guide on what and how much to post on each specific platform, tailoring content to unique audience expectations and platform features.

\subsection*{1. Instagram}
Instagram is a visual platform. Your content needs to be aesthetically pleasing.

\begin{coloredbox}
    \item \important{Content Types \& What to Post:}
    \begin{itemize}
        \item \important{Reels (Primary Focus):} Use the "Daily Reminders" and "Inspirational Storytelling" pillars. Short, visually striking videos with text overlays perform best.
        \item \important{Carousel Posts (Image Sets):} Perfect for "Educational Snippets." Create a 3-5 slide carousel that breaks down a topic. For example:
        \begin{itemize}
            \item Slide 1: A question (e.g., "What are the 5 Pillars of Islam?").
            \item Slide 2-4: Explain each point on a separate slide with a clean design.
            \item Slide 5: A call to action ("Save this post for your reference").
        \end{itemize}
        \item \important{Single Image Posts:} A beautiful image with a powerful Quranic verse or quote. Good for less frequent, high-impact posts.
        \item \important{Stories:} Use for daily engagement (polls, Q\&As) and to share other creators' content. This part is harder to automate and should be done manually for 15-20 minutes a day.
    \end{itemize}
    
    \item \important{Posting Frequency:}
    \begin{itemize}
        \item \important{Reels:} 1 per day.
        \item \important{Carousel/Image Post:} 3-4 times per week.
        \item \important{Stories:} 3-5 times per day (manual).
    \end{itemize}
\end{coloredbox}

\subsection*{2. TikTok}
TikTok is fast-paced and trend-driven. Your content needs to grab attention in the first 2 seconds.

\begin{coloredbox}
    \item \important{Content Types \& What to Post:}
    \begin{itemize}
        \item \important{Short Videos (Primary Focus):} This is TikTok's core. All four pillars work here, but they must be fast and engaging.
        \begin{itemize}
            \item \important{Daily Reminders:} Use trending (halal) sounds with your text on screen.
            \item \important{Educational Snippets:} Use the "Q\&A" format. A text bubble with a question appears, and you provide a quick answer.
            \item \important{Storytelling:} Keep stories under 60 seconds. Use quick cuts and engaging visuals.
        \end{itemize}
        \item \important{Stitches/Duets:} (Manual) A great way to interact with other videos and add your commentary.
    \end{itemize}
    
    \item \important{Posting Frequency:}
    \begin{itemize}
        \item \important{Videos:} 2-3 times per day. The algorithm rewards high volume.
    \end{itemize}
\end{coloredbox}

\subsection*{3. YouTube}
YouTube serves two purposes: short-form discovery (Shorts) and long-form education (Videos).

\begin{coloredbox}
    \item \important{Content Types \& What to Post:}
    \begin{itemize}
        \item \important{Shorts (For Discovery):}
        \begin{itemize}
            \item Use your best-performing Reels and TikToks here.
            \item "Daily Reminders" and quick "Educational Snippets" are perfect.
            \item \important{Goal:} Attract new subscribers who find you through the Shorts feed.
        \end{itemize}
        
        \item \important{Long-Form Videos (For Authority \& Monetization):}
        \begin{itemize}
            \item This is where you build authority. Create 5-10 minute videos.
            \item \important{Content Ideas:}
            \begin{itemize}
                \item "A Deep Dive into Surah Al-Fatiha"
                \item "The Full Story of Prophet Musa (AS)"
                \item "10 Steps to Improve Your Salah (Prayer)"
                \item Compilations of your "Daily Reminders" from the week.
            \end{itemize}
            \item \important{SEO is Key:} Use clear, searchable titles and detailed descriptions with keywords.
        \end{itemize}
    \end{itemize}
    
    \item \important{Posting Frequency:}
    \begin{itemize}
        \item \important{Shorts:} 1 per day.
        \item \important{Long-Form Videos:} 1-2 times per week. Quality is much more important than quantity here.
    \end{itemize}
\end{coloredbox}

\subsection*{4. Facebook}
Facebook is community-oriented. Content that sparks discussion and sharing does well.

\begin{coloredbox}
    \item \important{Content Types \& What to Post:}
    \begin{itemize}
        \item \important{Reels:} Cross-post your Instagram Reels here. Facebook's audience can be slightly older, so inspirational and straightforward reminders work very well.
        \item \important{Image/Text Posts:} Facebook is still very effective for text. Post a thoughtful question or a reflection related to a Quranic verse and ask for people's opinions in the comments. This is great for engagement.
        \item \important{Link Sharing:} Share links to your long-form YouTube videos to drive traffic.
        \item \important{Facebook Groups:} (Manual) Find relevant Islamic groups and share your content there (if the group rules permit). This is a powerful way to find your target audience.
    \end{itemize}
    
    \item \important{Posting Frequency:}
    \begin{itemize}
        \item \important{Reels/Videos:} 1 per day.
        \item \important{Image/Text Post:} 1 per day.
    \end{itemize}
\end{coloredbox}

\end{document}